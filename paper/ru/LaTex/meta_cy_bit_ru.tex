\documentclass[12pt,a4paper]{article}
\usepackage[utf8]{inputenc}
\usepackage[russian]{babel}
\usepackage{amsmath,amssymb}
\usepackage{geometry}
\geometry{margin=2.5cm}
\usepackage{hyperref}

\title{Мета-квантовые вычисления на многообразиях Калаби--Яу.\\ CYbit and MetaCYbit}
\author{Евгений Монахов \\ LCC "VOSCOM ONLINE" Research Initiative \\ 
\href{https://orcid.org/0009-0003-1773-5476}{ORCID: 0009-0003-1773-5476}}
\date{2025}

\begin{document}
\maketitle

\begin{abstract}
В работе вводится новая парадигма вычислений, основанная на многообразиях Калаби--Яу (CY).
Показан путь от классических битов к квантовым кубитам, далее к кудитам,
и затем к CYбитам — квантовым состояниям, определённым на многообразиях Калаби--Яу.
Наконец, вводится понятие Мета-CYбита — функционала, действующего на целые гильбертовы пространства CYбитов.
Приведены математические определения, оценки масштабирования и предложен план исследований
для дальнейшего изучения этой модели вычислений.
\end{abstract}

\section{Введение}
Классические вычисления основаны на битах, принимающих значения $0$ и $1$.  
Квантовые вычисления расширяют это понятие до кубитов — суперпозиций в двумерном гильбертовом пространстве.  
Кудиты обобщают это на размерность $d$, позволяя состояниям существовать в $\mathbb{C}^d$.  
Мы предлагаем сделать следующий шаг, введя CYбиты — состояния, определённые на многообразиях Калаби--Яу, 
и Мета-CYбиты, работающие на уровне функциональных пространств.  

\section{От битов к кубитам}
Классический бит:
\[
b \in \{0,1\}.
\]
Квантовый бит (кубит):
\[
|\psi\rangle = \alpha |0\rangle + \beta |1\rangle, \quad \alpha,\beta \in \mathbb{C}, \; |\alpha|^2+|\beta|^2=1.
\]

\section{Кудиты}
Обобщение на размерность $d$:
\[
|\psi\rangle = \sum_{i=0}^{d-1} \alpha_i |i\rangle, \quad \alpha_i \in \mathbb{C}, \quad \sum_{i=0}^{d-1}|\alpha_i|^2=1.
\]

\section{CYбиты}
Пусть $M$ — многообразие Калаби--Яу комплексной размерности $k$.  
Определим CYбит как:
\[
\psi(x) \in L^2(M,\mathbb{C}^d).
\]
Таким образом, носителями информации являются квантовые состояния на CY-геометрии.  

Лапласиан:
\[
(Lf)(p_i) = \sum_{j:(i,j)\in E} w_{ij}(f(p_i)-f(p_j)),
\]
спектр которого, как предполагается, отражает числа Ходжа $(h^{1,1},h^{2,1})$ 
и эйлерову характеристику $\chi(M)$.  

\section{Мета-CYбиты}
Определим Мета-CYбиты как функционалы на CY-гильбертовых пространствах:
\[
\Psi \in L^2(\mathcal{H}_{CY}, \mathcal{D}\psi),
\]
где $\mathcal{H}_{CY}$ — гильбертово пространство состояний CYбитов.  
Таким образом, они представляют вычисления над пространствами квантовых состояний.  

\section{Масштабирование вычислительной ёмкости}
Для $n$ носителей информации:
\begin{itemize}
\item Классические биты: $2^n$ состояний.  
\item Кубиты: $2^n$-мерное гильбертово пространство.  
\item Кудиты ($d$ уровней): $d^n$ состояний.  
\item CYбиты (размерность CY = $m$, локальное $d$-уровневое состояние): 
\[
\sim (m^d)^n \quad \text{эффективных состояний}.
\]
\item Мета-CYбиты: рост масштабируется как функциональные пространства, 
что приводит к сверхэкспоненциальному увеличению.  
\end{itemize}

\section{Обсуждение и направления исследований}
\begin{enumerate}
\item Спектральный анализ: связь спектра лапласиана и топологии CY.  
\item Вычислительная ёмкость: формулы, связывающие её с числами Ходжа.  
\item Коррекция ошибок: построение квантовых кодов на основе CY-гомологии.  
\item Динамика: время как поток Риччи на метриках CY,
\[
\frac{\partial g_{i\bar{j}}}{\partial \tau} = - \mathrm{Ric}(g)_{i\bar{j}}.
\]
\end{enumerate}

\section{Заключение}
Мета-CY вычисления представляют собой концептуальный скачок 
от кубитов к носителям информации, определённым на многообразиях, 
и далее — к функциональному уровню Мета-CYбитов.  
Эта рамка открывает широкий круг математических, физических 
и вычислительных задач.  

\section*{Цитирование (BibTeX - EN)}
\begin{verbatim}
@misc{CY_meta_quantum_2025,
  author       = {Evgeny Monakhov and LCC "VOSCOM ONLINE" Research Initiative},
  title        = {Meta-Quantum Computing on Calabi--Yau Manifolds},
  year         = {2025},
  publisher    = {Zenodo},
  doi          = {10.5281/zenodo.17050352},
  url          = {https://doi.org/10.5281/zenodo.17050352},
  orcid        = {0009-0003-1773-5476},
  url_orcid    = {https://orcid.org/0009-0003-1773-5476},
  organization = {https://voscom.online/}
}
\end{verbatim}

\end{document}
