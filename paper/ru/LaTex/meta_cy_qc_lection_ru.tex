\documentclass[12pt,a4paper]{article}
\usepackage[utf8]{inputenc}
\usepackage[T2A]{fontenc}
\usepackage[russian]{babel}
\usepackage{amsmath,amssymb}
\usepackage{hyperref}
\usepackage{geometry}
\geometry{margin=2.5cm}

\title{Мета-квантовые вычисления на многообразиях Калаби--Яу. Лекция.\\
от кудитов к CYбитам и Мета-CYбитам}
\author{[Евгений Монахов] \\ LCC "VOSCOM ONLINE" Research Initiative \\ https://orcid.org/0009-0003-1773-5476}
\date{Сентябрь 2025}

\begin{document}
\maketitle

\section*{Вступление}
Добрый день, коллеги. Сегодня я расскажу о концепции, которая расширяет 
современное понимание квантовых вычислений. 
Мы начнём с привычного --- бита и кубита --- и шаг за шагом поднимемся 
к новой вычислительной единице: \emph{CYбиту}, основанному на многообразиях 
Калаби--Яу. Затем мы сделаем ещё один шаг и введём \emph{Мета-CYбит} --- 
объект, который живёт в пространстве волновых функционалов. 
Таким образом, мы получаем систему, где квантовые принципы 
«вшиты» в саму структуру вычислений.

\section{От бита к кубиту}
Классический бит определяется как
\[
x \in \{0,1\}.
\]
Кубит описывается как суперпозиция:
\[
|\psi\rangle = \alpha|0\rangle + \beta|1\rangle, \qquad 
|\alpha|^2 + |\beta|^2 = 1.
\]
$n$ кубитов образуют пространство размерности $2^n$.

\section{Кудиты}
Обобщение кубита до $d$ измерений:
\[
|\psi\rangle = \sum_{i=1}^d \alpha_i |i\rangle, 
\qquad \sum_{i=1}^d |\alpha_i|^2 = 1.
\]
Кудиты позволяют хранить больше информации в одной ячейке 
и сокращать глубину квантовых схем.

\section{CYбит}
Пусть $M$ --- многообразие Калаби--Яу размерности $k$.
Мы определяем CYбит как сечение:
\[
\psi: M \to \mathbb{C}^d, \quad \psi(p)=(\psi_1(p),\dots,\psi_d(p)).
\]
Скалярное произведение:
\[
\langle \psi,\phi \rangle = \int_M \psi^\dagger(p)\phi(p)\, d\mu(p).
\]
Таким образом,
\[
\mathcal{H}_{CY} = L^2(M,\mathbb{C}^d).
\]

\section{CYlink и CYgluon}
Связь между CYбитами:
\[
H = \sum_{(p,q)\in E} w_{pq} e^{i\phi_{pq}} 
U_{pq} \otimes |p\rangle\langle q| + \text{h.c.}
\]
Параметры:
\begin{itemize}
\item $w_{pq}$ --- вес связи (метрика CY),
\item $\phi_{pq}$ --- фаза (интеграл Berry),
\item $U_{pq}$ --- оператор перехода.
\end{itemize}
Мы также вводим \emph{CYgluon} --- объект, описывающий взаимодействие связей:
\[
\hat{\mathcal{G}}(L_1,L_2) \sim g \int \Psi[\psi]\,F[L_1,L_2;\psi]\,\mathcal{D}\psi.
\]

\section{Мета-CYбит}
Ключевой шаг --- повторное квантование.
CYбит живёт в $\mathcal{H}_{CY}$.
Теперь вводим \emph{волновой функционал}:
\[
\Psi[\psi] \in L^2(\mathcal{H}_{CY},\mathcal{D}\psi).
\]
Нормировка:
\[
\int |\Psi[\psi]|^2 \,\mathcal{D}\psi = 1.
\]
Эволюция описывается функциональным уравнением Шрёдингера:
\[
i\hbar \frac{\partial}{\partial t} \Psi[\psi] 
= \hat{\mathcal{H}} \Psi[\psi].
\]

\section{Сравнение уровней}
\begin{tabular}{|c|c|c|}
\hline
Уровень & Объект & Пространство состояний \\
\hline
0 & Бит & $\{0,1\}$ \\
1 & Кубит/Кудит & $\mathbb{C}^d$ \\
2 & CYбит & $\mathcal{H}_{CY} = L^2(M,\mathbb{C}^d)$ \\
3 & Мета-CYбит & $\mathcal{H}_{Meta} = L^2(\mathcal{H}_{CY},\mathcal{D}\psi)$ \\
\hline
\end{tabular}

\section{Вычислительный потенциал}
Примерные оценки:
\begin{itemize}
\item Классический ПК: $10^{12}$ операций/с.
\item 50 кубитов: $\sim 10^{15}$ амплитуд.
\item 10 CYбитов (3D, $m=10$): $10^{30}$ амплитуд.
\item 10 CYбитов (6D, $m=10$): $10^{60}$ амплитуд.
\item Мета-CYбит: суперэкспоненциальное пространство функционалов.
\end{itemize}

\section{Последствия}
\textbf{Математика:} вычислительная геометрия Калаби--Яу. \\
\textbf{Физика:} новые применения CY вне космологии. \\
\textbf{Информатика:} новая парадигма мета-квантовых вычислений. \\
\textbf{ИИ:} потенциальные новые архитектуры для обучения и моделирования.

\section*{Заключение}
Мы прошли путь: бит $\to$ кубит $\to$ кудит $\to$ CYбит $\to$ Мета-CYбит.  
Эта иерархия открывает новый класс вычислительных моделей, 
где квантовые принципы встроены в саму структуру архитектуры. 
Дальнейшие исследования могут привести к созданию устройств, 
работающих в пространствах, которые мы пока только начинаем постигать.

\section*{Citation (BibTeX - EN)}
\begin{verbatim}
@misc{CY_meta_quantum_2025,
  author       = {Evgeny Monakhov and LCC "VOSCOM ONLINE" Research Initiative},
  title        = {Meta-Quantum Computing on Calabi--Yau Manifolds},
  year         = {2025},
  publisher    = {Zenodo},
  doi          = {10.5281/zenodo.17050352},
  url          = {https://doi.org/10.5281/zenodo.17050352}
  orcid		   = {0009-0003-1773-5476}
  url_orcid     = {https://orcid.org/0009-0003-1773-5476}
  organization = {https://voscom.online/}
}
\end{verbatim}

\end{document}
