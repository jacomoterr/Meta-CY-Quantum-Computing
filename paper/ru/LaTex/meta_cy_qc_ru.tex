\documentclass[12pt,a4paper]{article}
\usepackage[utf8]{inputenc}
\usepackage[T2A]{fontenc}
\usepackage[russian]{babel}
\usepackage{amsmath,amssymb}
\usepackage{hyperref}
\usepackage{geometry}
\geometry{margin=2.5cm}

\title{Спектральные графы на многообразиях Калаби--Яу: \\
Исследовательская гипотеза и программа}
\author{[Ваше Имя] \\ LCC "VOSCOM ONLINE" Research Initiative \\ https://orcid.org/0009-0003-1773-5476}
\date{Сентябрь 2025}

\begin{document}
\maketitle

\begin{abstract}
В данной работе представлена исследовательская гипотеза и программа, 
посвящённые спектральным свойствам графов, построенных на основе 
многообразий Калаби--Яу (CY). 
Основное утверждение состоит в том, что дискретные лапласианы на CY-графах 
отражают нетривиальную топологическую и геометрическую информацию, 
связанную с числами Ходжа и характеристикой Эйлера. 
Предлагается систематическое исследование спектральных моделей, вдохновлённых CY, 
как шаг к построению теории CY-битов и их мета-квантовых расширений. 
Хотя в данной работе отсутствуют численные эксперименты, 
она задаёт теоретическую основу и намечает подробный план исследовательских задач.
\end{abstract}

\section{Введение и мотивация}
Изучение вычислений в многомерных структурах 
может дать новые идеи для более простых моделей. 
Многообразия Калаби--Яу хорошо известны из теории струн и комплексной геометрии 
как богатые математические структуры с жёсткими топологическими ограничениями. 
Мы предполагаем, что спектральная теория графов на дискретизациях CY-пространств 
может выявить вычислительные особенности, недоступные в обычных евклидовых условиях.  
Настоящий текст следует рассматривать как программное исследовательское сообщение: 
он формулирует гипотезы и предлагает систематический план их проверки.

\section{Определения и постановка задачи}
Пусть $M$ — многообразие Калаби--Яу комплексной размерности $k$.  
Рассмотрим дискретизацию $\{p_1,\dots,p_N\}\subset M$ 
с рёбрами, заданными по правилу $k$ ближайших соседей.  
Дискретный лапласиан определяется как
\[
(Lf)(p_i) = \sum_{j:(i,j)\in E} w_{ij}(f(p_i)-f(p_j)),
\]
где $w_{ij}$ зависит от CY-метрики.
Нас интересует спектр $\{\lambda_\alpha\}$ оператора $L$ 
и его асимптотика при $N \to \infty$.

\section{Центральная гипотеза}
Мы предполагаем, что:
\begin{itemize}
\item Распределение собственных значений $L$ при $N \to \infty$ 
стремится к спектру оператора Лапласа–Бельтрами на $M$.  
\item Спектральные разрывы и вырождения содержат информацию о числах Ходжа 
$(h^{1,1}, h^{2,1})$ многообразия $M$.  
\item Эти спектральные характеристики могут быть использованы для определения 
\emph{вычислительной ёмкости} $\mathcal{C}(M)$ 
в CY-вдохновлённых квантовых моделях.  
\end{itemize}

\section{Программа исследований}
Предлагаемая программа исследований структурирована по этапам:

\subsection{Этап 1: Методы дискретизации}
\begin{enumerate}
\item Разработать численные дискретизации простых CY-пространств: 
2-тор $T^2$, 3-тор $T^3$ и поверхность K3.  
\item Реализовать как случайное распределение точек, так и решётчатые сетки.  
\item Определить веса рёбер с использованием приближённых риккитиплых метрик 
(например, по алгоритму Дональдсона).  
\end{enumerate}

\subsection{Этап 2: Спектральный анализ}
\begin{enumerate}
\item Вычислить спектры дискретных лапласианов при росте $N$.  
\item Изучить свойства сходимости к спектру непрерывного оператора.  
\item Проанализировать плотность спектра, разрывы и вырождения.  
\end{enumerate}

\subsection{Этап 3: Топологические проверки}
\begin{enumerate}
\item Сравнить структуру вырождений с известными числами Ходжа.  
\item Исследовать корреляции между низколежащими собственными значениями 
и характеристикой Эйлера.  
\item Проверить устойчивость этих соответствий 
при различных схемах дискретизации.  
\end{enumerate}

\subsection{Этап 4: Вычислительная интерпретация}
\begin{enumerate}
\item Определить и уточнить понятие \emph{вычислительной ёмкости} $\mathcal{C}(M)$ 
как функции спектральных инвариантов.  
\item Интерпретировать $\mathcal{C}(M)$ в терминах возможных кодировок квантовых состояний.  
\item Ввести аналогию с кубитами и предложить CY-биты 
как вычислительные примитивы.  
\end{enumerate}

\subsection{Этап 5: К мета-CYбитам}
\begin{enumerate}
\item Расширить методику на более сложные CY-3, 
например квинтику в $\mathbb{CP}^4$.  
\item Изучить устойчивость спектра при деформациях комплексной структуры.  
\item Определить мета-CYбиты как составные структуры, построенные из семейств CY-битов.  
\end{enumerate}

\section{Ожидаемые результаты}
Подтверждение этих гипотез позволит установить количественную связь 
между топологией CY и вычислительной ёмкостью. 
Это станет первым шагом к CY-битам и мета-CYбитам, 
открывая основу для нового класса мета-квантовых вычислительных моделей.  

\section*{Благодарности}
Автор благодарит инициативу VOSCOM ONLINE Research 
за предоставленную концептуальную основу.

\begin{thebibliography}{9}
\bibitem{candelas1985}
Candelas и др., ``Vacuum configurations for superstrings,'' 
Nucl. Phys. B, 1985.
\bibitem{exner2008}
Exner, P., ``Quantum graphs: An introduction,'' 
Annals of Physics, 2008.
\bibitem{belhaj2014}
Belhaj, A. и др., ``Qubits from Calabi--Yau manifolds and toric geometry,'' 
arXiv:1408.3952, 2014.
\bibitem{donaldson2005}
Donaldson, S.K., ``Numerical results on Calabi--Yau metrics,'' 
Class. Quantum Grav., 2005.
\bibitem{chung1997}
Chung, F., ``Spectral Graph Theory,'' 
CBMS Regional Conference Series in Mathematics, 1997.
\end{thebibliography}

\section*{Citation (BibTeX - EN)}
\begin{verbatim}
@misc{CY_meta_quantum_2025,
  author       = {Evgeny Monakhov and LCC "VOSCOM ONLINE" Research Initiative},
  title        = {Meta-Quantum Computing on Calabi--Yau Manifolds},
  year         = {2025},
  publisher    = {Zenodo},
  doi          = {10.5281/zenodo.17050352},
  url          = {https://doi.org/10.5281/zenodo.17050352}
  orcid		   = {0009-0003-1773-5476}
  url_orcid     = {https://orcid.org/0009-0003-1773-5476}
  organization = {https://voscom.online/}
}
\end{verbatim}

\end{document}
