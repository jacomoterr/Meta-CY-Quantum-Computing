\documentclass[12pt,a4paper]{article}
\usepackage[utf8]{inputenc}
\usepackage[T2A]{fontenc}
\usepackage[russian]{babel}
\usepackage{amsmath,amssymb}
\usepackage{hyperref}
\usepackage{geometry}
\geometry{margin=2.5cm}

\title{Meta-CY квантовые вычисления: спектральные графы на многообразиях Калаби--Яу}
\author{Евгений Монахов \\ LCC "VOSCOM ONLINE" Research Initiative \\ https://orcid.org/0009-0003-1773-5476}
\date{2025}

\begin{document}
\maketitle

\section*{1. Центральная идея}
В данной работе вводится вычислительная модель, в которой 
носителями информации являются не только кубиты или кудиты, 
но и волновые функции на многообразиях Калаби--Яу (CY).  
Подход объединяет геометрию CY с методами спектральной теории графов.  

\subsection*{1.1 CY-бит}
Для многообразия Калаби--Яу $M$ комплексной размерности $k$ 
\emph{CY-бит} определяется как
\begin{equation}
\psi \in L^2(M, \mathbb{C}^d),
\end{equation}
где $d$ — локальная размерность (аналог кудита).  

\subsection*{1.2 CY-графы}
Система CY-битов может быть представлена графом $G=(V,E)$, 
вложенным в $M$.  
Вес рёбер определяется расстояниями и топологическими циклами:
\begin{equation}
w_{ij} = f(\mathrm{dist}_M(p_i,p_j), \mathrm{Top}(M)).
\end{equation}

\subsection*{1.3 Спектральный лапласиан}
Лапласиан на таком графе кодирует 
как метрические, так и топологические свойства $M$.  
Собственные значения и векторы описывают 
возможные энергетические состояния и переходы.  

\section*{2. Мотивация}
\begin{itemize}
  \item Классические компьютеры ограничены $\sim 10^{12}$ операций/с.
  \item Квантовые компьютеры: $2^n$ состояний за счёт кубитов.
  \item Кудиты: $d^n$ состояний при $d>2$.
  \item CY-биты: экспоненциальное расширение за счёт структуры CY.
\end{itemize}

\section*{3. Формальная структура}
\begin{itemize}
  \item Гильбертово пространство: $L^2(M, \mathbb{C}^d)$.
  \item Представление в виде графа: оператор смежности $A$.
  \item Гамильтониан:
  \[
  H = -\Delta_{CY} + V + H_{\text{int}}
  \]
  где $\Delta_{CY}$ — лапласиан на CY.
\end{itemize}

\section*{4. Потенциал масштабирования}
\begin{center}
\begin{tabular}{|c|c|c|}
\hline
Система & Локальная размерность & $n=10$ узлов \\
\hline
Кубиты (2D) & 2 & $2^{10} \sim 10^3$ \\
Кудиты ($d=10$) & 10 & $10^{10}$ \\
CY-3D ($m=10$) & $10^3$ & $10^{30}$ \\
CY-6D ($m=10$) & $10^6$ & $10^{60}$ \\
\hline
\end{tabular}
\end{center}

\section*{5. Дорожная карта исследований}
\begin{enumerate}
  \item Теоретика: CY-биты, CY-связи, лапласианы.
  \item Математика: mirror symmetry, инварианты, топология CY.
  \item Симуляции: спектральный анализ на тораx $T^2$, $T^3$.
  \item Эксперименты: прототипы на фотонных или ионных системах ($d=3-5$).
  \item Долгосрочная цель: масштабируемый CY-квантовый компьютер.
\end{enumerate}

\section*{6. Заключение}
Предложен новый подход в квантовой информатике: 
\emph{Meta-CY квантовые вычисления}.  
Он объединяет геометрию и топологию CY с теорией спектральных графов, 
открывая возможность радикального расширения вычислительных мощностей.

\section*{Citation (BibTeX - EN)}
\begin{verbatim}
@misc{CY_meta_quantum_2025,
  author       = {Evgeny Monakhov and LCC "VOSCOM ONLINE" Research Initiative},
  title        = {Meta-Quantum Computing on Calabi--Yau Manifolds},
  year         = {2025},
  publisher    = {Zenodo},
  doi          = {10.5281/zenodo.17050352},
  url          = {https://doi.org/10.5281/zenodo.17050352}
  orcid		   = {0009-0003-1773-5476}
  url_orcid     = {https://orcid.org/0009-0003-1773-5476}
  organization = {https://voscom.online/}
}
\end{verbatim}

\end{document}