\documentclass[12pt,a4paper]{article}
\usepackage[utf8]{inputenc}
\usepackage[russian]{babel}
\usepackage{amsmath,amssymb}
\usepackage{geometry}
\geometry{margin=2.5cm}
\usepackage{hyperref}

\title{План исследований по развитию теории Meta-CY квантовых вычислений}
\author{Евгений Монахов \\ LCC "VOSCOM ONLINE" Research Initiative \\ 
\href{https://orcid.org/0009-0003-1773-5476}{ORCID: 0009-0003-1773-5476}}
\date{2025}

\begin{document}
\maketitle

\section*{Введение}
Представленная программа исследований направлена на систематическое развитие теории
Meta-CY квантовых вычислений, где носителями информации выступают состояния на многообразиях Калаби--Яу (CY),
а операции включают спектральные, топологические и функциональные преобразования.  

\section{Основные направления исследований}

\subsection{CY-Time: время как поток Риччи}
\begin{itemize}
\item Модель: 
\[
\frac{\partial g_{i\bar{j}}}{\partial \tau} = - \mathrm{Ric}(g)_{i\bar{j}}.
\]
\item Задача: описать время как проекцию внутренней динамики CY.  
\item Перспектива: построение предсказательных моделей эволюции.  
\end{itemize}

\subsection{Коды коррекции ошибок на основе CY-гомологии}
\begin{itemize}
\item Модель кодов:
\[
[[n,k,d]], \quad d = \min\{\text{размер нетривиального цикла}\}.
\]
\item Задача: построение новых семейств устойчивых квантовых кодов.  
\item Перспектива: обобщение торического кода на многомерные CY.  
\end{itemize}

\subsection{Спектральные свойства CY-графов}
\begin{itemize}
\item Дискретный лапласиан:
\[
(Lf)(p_i) = \sum_{j:(i,j)\in E} w_{ij}(f(p_i)-f(p_j)).
\]
\item Гипотеза: спектр отражает числа Ходжа $(h^{1,1},h^{2,1})$ и эйлерову характеристику $\chi(M)$.  
\item Перспектива: новая связь спектральной геометрии и алгебраической топологии.  
\end{itemize}

\subsection{Вычислительная ёмкость CY-многообразий}
\begin{itemize}
\item Определение:
\[
\mathcal{C}(M) = \log \dim_{\mathrm{eff}}(\mathcal{H}_{CY}).
\]
\item Гипотеза:
\[
\mathcal{C}(M) \sim f(h^{1,1}, h^{2,1}, \chi(M)).
\]
\item Перспектива: количественная теория вычислительной мощности CYбитов.  
\end{itemize}

\subsection{CYlinks и CYgluons}
\begin{itemize}
\item CYlink:
\[
H_{link}(i,j) = w_{ij}\,\psi^\dagger(p_i)\psi(p_j) + h.c.
\]
\item CYgluon:
\[
H_{gluon}((i,j),(k,l)) = g_{ijkl}\,\psi^\dagger(p_i)\psi(p_j)\psi^\dagger(p_k)\psi(p_l).
\]
\item Перспектива: построение новых архитектур вычислений с многоуровневыми взаимодействиями.  
\end{itemize}

\subsection{Мета-CYбиты как функционалы}
\begin{itemize}
\item Определение:
\[
\Psi \in L^2(\mathcal{H}_{CY}, \mathcal{D}\psi).
\]
\item Задача: описать вычисления на уровне пространств состояний.  
\item Перспектива: гипервычисления за пределами BQP.  
\end{itemize}

\subsection{Зеркальная симметрия как вычислительная двойственность}
\begin{itemize}
\item Задача: формализовать зеркальную симметрию как преобразование вычислений.  
\item Перспектива: алгоритмические двойственности и перевод задач между режимами.  
\end{itemize}

\subsection{Фазы в сетях CY-квантов}
\begin{itemize}
\item Задача: исследовать коллективное поведение сетей CYбитов.  
\item Перспектива: появление новых вычислительных фаз (аналог конденсата Бозе или сверхпроводимости).  
\end{itemize}

\section{Заключение}
Предложенный план охватывает как математические, так и вычислительные аспекты.
Его реализация позволит создать строгую теорию Meta-CY квантовых вычислений
и выявить новые архитектуры с уникальными свойствами устойчивости и мощности.  

\section*{Citation (BibTeX - EN)}
\begin{verbatim}
@misc{CY_links_gluons_2025,
  author       = {Evgeny Monakhov and LCC "VOSCOM ONLINE" Research Initiative},
  title        = {CYlinks and CYgluons: Interactions in Meta-Quantum Computing},
  year         = {2025},
  publisher    = {Zenodo},
  doi          = {10.5281/zenodo.17050353},
  url          = {https://doi.org/10.5281/zenodo.17050353},
  orcid        = {0009-0003-1773-5476},
  url_orcid    = {https://orcid.org/0009-0003-1773-5476},
  organization = {https://voscom.online/}
}
\end{verbatim}

\end{document}