\documentclass[12pt,a4paper]{article}
\usepackage[utf8]{inputenc}
\usepackage[russian]{babel}
\usepackage{amsmath,amssymb}
\usepackage{geometry}
\geometry{margin=2.5cm}
\usepackage{hyperref}

\title{CYlinks и CYgluons: взаимодействия в мета-квантовых вычислениях}
\author{Евгений Монахов \\ LCC "VOSCOM ONLINE" Research Initiative \\ 
\href{https://orcid.org/0009-0003-1773-5476}{ORCID: 0009-0003-1773-5476}}
\date{2025}

\begin{document}
\maketitle

\begin{abstract}
Предлагается формальное описание структур \emph{CYlinks} и \emph{CYgluons}, 
которые задают взаимодействия в рамках концепции Мета-CY квантовых вычислений.  
CYlinks представляют собой связи между CYбитами, определяемые вложениями в 
подпространства Калаби--Яу (CY), тогда как CYgluons описывают взаимодействия 
между самими связями, аналогично глюонам в квантовой хромодинамике (QCD).  
Приведены математические определения, операторные формализмы, 
план исследований и перспективы вычислительных приложений.  
\end{abstract}

\section{Введение}
В рамках Мета-CY квантовых вычислений CYбиты являются квантовыми состояниями, определёнными на многообразиях Калаби--Яу.  
Для построения вычислений необходимо их взаимодействие.  
Эти взаимодействия не ограничиваются парными связями, но могут сами образовывать 
сети более высокого порядка динамики.  
Мы называем такие структуры CYlinks (прямые связи) и CYgluons (взаимодействия между связями).  

\section{Определение CYlinks}
Пусть $M$ — многообразие Калаби--Яу, дискретизированное в виде графа $G=(V,E)$.  
Для двух CYбитов в точках $p_i,p_j \in M$ определим CYlink как оператор
\begin{equation}
H_{link}(i,j) = w_{ij} \, \psi^\dagger(p_i)\psi(p_j) + h.c.,
\end{equation}
с весом
\begin{equation}
w_{ij} = f(\mathrm{dist}_M(p_i,p_j), \;\mathcal{T}_{ij}),
\end{equation}
где $\mathcal{T}_{ij}$ кодирует топологические данные вложенного подпространства.

\subsection{Форма через графовый лапласиан}
Полный гамильтониан связей может быть записан как:
\begin{equation}
H_{links} = \sum_{(i,j)\in E} w_{ij} ( \psi^\dagger(p_i)\psi(p_j) + h.c. ).
\end{equation}
Это обобщает стандартные взаимодействия по смежности на CY-зависимые веса.

\section{CYgluons: взаимодействия между связями}
CYlinks сами могут взаимодействовать, образуя связи более высокого порядка.  
Определим оператор CYgluon, действующий на две связи $(i,j)$ и $(k,l)$:
\begin{equation}
H_{gluon}((i,j),(k,l)) = g_{ijkl} \, \psi^\dagger(p_i)\psi(p_j)\psi^\dagger(p_k)\psi(p_l),
\end{equation}
где
\begin{equation}
g_{ijkl} = g(\mathcal{T}_{ij}, \mathcal{T}_{kl}, \; \mathrm{Hom}(M)).
\end{equation}
Здесь $g_{ijkl}$ зависит от пересечений CY-подпространств и гомологических связей.  

\subsection{Полный гамильтониан}
Глобальный гамильтониан системы имеет вид:
\begin{equation}
H = \sum_i H_{CYbit}(i) + \sum_{(i,j)} H_{link}(i,j) + \sum_{(i,j),(k,l)} H_{gluon}((i,j),(k,l)).
\end{equation}

\section{План исследований}
\subsection{Этап I: формализм CYlink}
\begin{enumerate}
\item Определить явные $w_{ij}$ для простых CY (торы $T^n$, K3).  
\item Вычислить спектры гамильтонианов CYlinks.  
\item Установить связь $w_{ij}$ с топологией CY.  
\end{enumerate}

\subsection{Этап II: структуры CYgluons}
\begin{enumerate}
\item Ввести $g_{ijkl}$ на основе пересечений CY-подпространств.  
\item Проверить согласованность с калибровочными симметриями.  
\item Изучить аналогию с цветовыми зарядами QCD.  
\end{enumerate}

\subsection{Этап III: объединённая динамика}
\begin{enumerate}
\item Смоделировать сети CYбитов с CYlinks и CYgluons.  
\item Изучить устойчивость и свойства коррекции ошибок.  
\item Исследовать возникающие вычислительные фазы.  
\end{enumerate}

\section{Перспективы}
\begin{itemize}
\item \textbf{Вычислительная мощность:} CYlinks и CYgluons дают новые структуры взаимодействия, усиливающие выразительные возможности.  
\item \textbf{Коррекция ошибок:} связи CYgluon могут стабилизировать логические состояния через избыточность более высокого порядка.  
\item \textbf{Физическая аналогия:} формальное сходство с теориями калибровочных полей указывает на возможное расширение к CY-группам калибровки.  
\end{itemize}

\section{Заключение}
CYlinks и CYgluons расширяют модель Мета-CYбитов, вводя структурированные взаимодействия 
между CYбитами и между их связями.  
Это создаёт основу для новых квантовых вычислительных архитектур, 
опирающихся на геометрию Калаби--Яу.  

\section*{Цитирование (BibTeX - EN)}
\begin{verbatim}
@misc{CY_links_gluons_2025,
  author       = {Evgeny Monakhov and LCC "VOSCOM ONLINE" Research Initiative},
  title        = {CYlinks and CYgluons: Interactions in Meta-Quantum Computing},
  year         = {2025},
  publisher    = {Zenodo},
  doi          = {10.5281/zenodo.17050353},
  url          = {https://doi.org/10.5281/zenodo.17050353},
  orcid        = {0009-0003-1773-5476},
  url_orcid    = {https://orcid.org/0009-0003-1773-5476},
  organization = {https://voscom.online/}
}
\end{verbatim}

\end{document}
