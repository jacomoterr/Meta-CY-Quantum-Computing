\documentclass[12pt,a4paper]{article}
\usepackage[utf8]{inputenc}
\usepackage{amsmath,amssymb}
\usepackage{geometry}
\geometry{margin=2.5cm}

\title{Meta-CY Quantum Computing: Spectral Graphs on Calabi--Yau Manifolds}
\author{Evgeny Monakhov \\ VOSCOM Research Initiative}
\date{2025}

\begin{document}
\maketitle

\section*{1. Central Idea}
This work introduces a computational framework in which 
information carriers are defined not only as qubits or qudits 
but as wavefunctions on Calabi--Yau (CY) manifolds. 
The approach combines CY geometry with spectral graph theory.

\subsection*{1.1 CYbit}
For a Calabi--Yau manifold $M$ of complex dimension $k$, 
a \emph{CYbit} is defined as
\begin{equation}
\psi \in L^2(M, \mathbb{C}^d),
\end{equation}
with $d$ being the local dimension (qudit-like). 

\subsection*{1.2 CY Graphs}
A system of CYbits can be represented by a graph $G=(V,E)$ 
embedded in $M$. Edge weights are determined by distances 
and topological cycles:
\begin{equation}
w_{ij} = f(\mathrm{dist}_M(p_i,p_j), \mathrm{Top}(M)).
\end{equation}

\subsection*{1.3 Spectral Laplacian}
The Laplacian on such a graph encodes 
both metric and topological properties of $M$. 
Eigenvalues and eigenvectors define 
possible energy states and transitions.

\section*{2. Motivation}
\begin{itemize}
  \item Classical computers: bounded by $10^{12}$ ops/s.
  \item Quantum computers: $2^n$ states from qubits.
  \item Qudits: $d^n$ states with $d > 2$.
  \item CYbits: exponential extension via CY structure.
\end{itemize}

\section*{3. Formal Structure}
\begin{itemize}
  \item Hilbert space: $L^2(M, \mathbb{C}^d)$.
  \item Graph representation: adjacency operator $A$.
  \item Hamiltonian:
  \[
  H = -\Delta_{CY} + V + H_{\text{int}}
  \]
  where $\Delta_{CY}$ is the Laplacian on CY.
\end{itemize}

\section*{4. Scaling Potential}
\begin{center}
\begin{tabular}{|c|c|c|}
\hline
System & Local dimension & $n=10$ sites \\
\hline
Qubits (2D) & 2 & $2^{10} \sim 10^3$ \\
Qudits ($d=10$) & 10 & $10^{10}$ \\
CY-3D ($m=10$) & $10^3$ & $10^{30}$ \\
CY-6D ($m=10$) & $10^6$ & $10^{60}$ \\
\hline
\end{tabular}
\end{center}

\section*{5. Research Roadmap}
\begin{enumerate}
  \item Theoretical definitions: CYbits, CYlinks, Laplacians.
  \item Mathematics: mirror symmetry, invariants, topology of CY.
  \item Simulations: spectral numerics for torus $T^2$, $T^3$.
  \item Experimental: prototypes with $d=3-5$ photonic or ion states.
  \item Long-term: scalable CY quantum computation.
\end{enumerate}

\section*{6. Conclusion}
This proposal formulates a new paradigm of quantum information: 
\emph{Meta-CY Quantum Computing}. 
It unites CY geometry, topology, and spectral graphs 
to vastly extend computational capacity.

\end{document}
