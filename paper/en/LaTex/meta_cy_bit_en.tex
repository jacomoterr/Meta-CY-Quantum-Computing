\documentclass[12pt,a4paper]{article}
\usepackage[utf8]{inputenc}
\usepackage{amsmath,amssymb}
\usepackage{geometry}
\geometry{margin=2.5cm}
\usepackage{hyperref}

\title{Meta-Quantum Computing on Calabi--Yau Manifolds \\ CYbit and MetaCYbit}
\author{Evgeny Monakhov \\ LCC "VOSCOM ONLINE" Research Initiative \\ 
\href{https://orcid.org/0009-0003-1773-5476}{ORCID: 0009-0003-1773-5476}}
\date{2025}

\begin{document}
\maketitle

\begin{abstract}
This paper introduces a novel computational paradigm based on Calabi--Yau (CY) manifolds.
We present a progression from classical bits to quantum qubits, higher-dimensional qudits,
and then to CYbits --- quantum states defined on Calabi--Yau manifolds. 
Finally, we introduce Meta-CYbits, functionals acting on entire CY Hilbert spaces. 
We derive mathematical definitions, scaling estimates, and propose a roadmap 
for further exploration of this new model of computation. 
\end{abstract}

\section{Introduction}
Classical computation is based on bits, taking discrete values $0$ and $1$. 
Quantum computation extends this to qubits, superpositions in a two-dimensional Hilbert space.
Qudits generalize to dimension $d$, allowing states in $\mathbb{C}^d$. 
We propose to further extend this hierarchy by defining CYbits --- 
states defined on Calabi--Yau manifolds --- and Meta-CYbits, 
which operate at the level of functional spaces. 

\section{From Bits to Qubits}
A classical bit:
\[
b \in \{0,1\}.
\]
A quantum bit (qubit):
\[
|\psi\rangle = \alpha |0\rangle + \beta |1\rangle, \quad \alpha,\beta \in \mathbb{C}, \; |\alpha|^2+|\beta|^2=1.
\]

\section{Qudits}
Generalization to dimension $d$:
\[
|\psi\rangle = \sum_{i=0}^{d-1} \alpha_i |i\rangle, \quad \alpha_i \in \mathbb{C}, \quad \sum_{i=0}^{d-1}|\alpha_i|^2=1.
\]

\section{CYbits}
Let $M$ be a Calabi--Yau manifold of complex dimension $k$.  
Define a CYbit as:
\[
\psi(x) \in L^2(M,\mathbb{C}^d).
\]
Thus, information carriers are quantum states over CY geometry.  

The Laplacian operator:
\[
(Lf)(p_i) = \sum_{j:(i,j)\in E} w_{ij}(f(p_i)-f(p_j)),
\]
encodes spectral and topological properties of $M$, 
with eigenvalues conjectured to reflect Hodge numbers $(h^{1,1},h^{2,1})$ 
and Euler characteristic $\chi(M)$.

\section{Meta-CYbits}
We define Meta-CYbits as functionals over CY Hilbert spaces:
\[
\Psi \in L^2(\mathcal{H}_{CY}, \mathcal{D}\psi),
\]
where $\mathcal{H}_{CY}$ is the Hilbert space of CYbit states.
These represent computations over spaces of quantum states themselves.

\section{Scaling of Computational Capacity}
For $n$ information carriers:
\begin{itemize}
\item Classical bits: $2^n$ states.  
\item Qubits: $2^n$-dimensional Hilbert space.  
\item Qudits ($d$-level): $d^n$ states.  
\item CYbits (dimension $m$ CY manifold, local $d$-level): 
\[
\sim (m^d)^n \quad \text{effective states}.
\]
\item Meta-CYbits: scaling extends to functional spaces, potentially 
beyond hyper-exponential growth.
\end{itemize}

\section{Discussion and Research Directions}
\begin{enumerate}
\item Spectral analysis: relation between Laplacian spectrum and CY topology.  
\item Computational capacity: formulas linking capacity to Hodge numbers.  
\item Error correction: construction of quantum codes from CY homology.  
\item Dynamics: time as Ricci flow on CY metrics,
\[
\frac{\partial g_{i\bar{j}}}{\partial \tau} = - \mathrm{Ric}(g)_{i\bar{j}}.
\]
\end{enumerate}

\section{Conclusion}
Meta-CY Quantum Computing offers a conceptual leap 
from qubits to manifold-based information carriers, 
and ultimately to functional-level Meta-CYbits. 
This framework opens a wide range of mathematical, physical, 
and computational problems.

\section*{Citation (BibTeX - EN)}
\begin{verbatim}
@misc{CY_meta_quantum_2025,
  author       = {Evgeny Monakhov and LCC "VOSCOM ONLINE" Research Initiative},
  title        = {Meta-Quantum Computing on Calabi--Yau Manifolds},
  year         = {2025},
  publisher    = {Zenodo},
  doi          = {10.5281/zenodo.17050352},
  url          = {https://doi.org/10.5281/zenodo.17050352},
  orcid        = {0009-0003-1773-5476},
  url_orcid    = {https://orcid.org/0009-0003-1773-5476},
  organization = {https://voscom.online/}
}
\end{verbatim}

\end{document}
