\documentclass[12pt,a4paper]{article}
\usepackage[utf8]{inputenc}
\usepackage{amsmath,amssymb}
\usepackage{hyperref}
\usepackage{geometry}
\usepackage{verbatim} % for citation block
\geometry{margin=2.5cm}

\title{Spectral Graphs on Calabi--Yau Manifolds: \\
A Research Hypothesis and Program}
\author{[Your Name] \\ LCC "VOSCOM ONLINE" Research Initiative \\ https://orcid.org/0009-0003-1773-5476}
\date{September 2025}

\begin{document}
\maketitle

\begin{abstract}
This paper presents a research hypothesis and program concerning 
the spectral properties of graphs derived from Calabi--Yau (CY) manifolds. 
The central claim is that discrete Laplacians on CY graphs capture 
nontrivial topological and geometric information 
related to Hodge numbers and Euler characteristics. 
We propose a systematic study of CY-inspired spectral models 
as a step toward building the theory of CYbits and 
their meta-quantum extensions. 
Although the present work does not include numerical experiments, 
it establishes the theoretical framework and outlines 
a detailed roadmap of research tasks.
\end{abstract}

\section{Introduction and Motivation}
Studying computation in higher-dimensional frameworks 
can provide new insights into lower-dimensional models. 
Calabi--Yau manifolds are well known from string theory 
and complex geometry as rich mathematical structures 
with highly constrained topology. 
We hypothesize that spectral graph theory on discretizations of CY spaces 
can reveal computational features not accessible in ordinary Euclidean settings.
This document should be read as a programmatic research note: 
it makes conjectures and proposes a systematic plan for verification.

\section{Definitions and Setup}
Let $M$ be a Calabi--Yau manifold of complex dimension $k$.  
We define a discretization $\{p_1,\dots,p_N\}\subset M$ 
with edges given by $k$-nearest neighbors.  
The discrete Laplacian is
\[
(Lf)(p_i) = \sum_{j:(i,j)\in E} w_{ij}(f(p_i)-f(p_j)),
\]
where $w_{ij}$ depends on the CY metric.
We are interested in the spectrum $\{\lambda_\alpha\}$ of $L$ 
and its large-$N$ asymptotics.

\section{Central Hypothesis}
We conjecture that:
\begin{itemize}
\item The eigenvalue distribution of $L$ converges, as $N\to\infty$, 
to the spectrum of the Laplace--Beltrami operator on $M$.  
\item Spectral gaps and degeneracies encode information about 
the Hodge numbers $(h^{1,1}, h^{2,1})$ of $M$.  
\item These spectral features can be used to define 
a \emph{computational capacity} $\mathcal{C}(M)$ 
for CY-inspired quantum models.  
\end{itemize}

\section{Research Program}
The proposed research program is structured into several stages:

\subsection{Stage 1: Discretization Methods}
\begin{enumerate}
\item Develop numerical discretizations of simple Calabi--Yau spaces: 
2-torus $T^2$, 3-torus $T^3$, and the K3 surface.  
\item Implement both random sampling and lattice-based point sets.  
\item Define edge weights using approximate Ricci-flat metrics 
(as in Donaldson’s algorithm).  
\end{enumerate}

\subsection{Stage 2: Spectral Analysis}
\begin{enumerate}
\item Compute spectra of discrete Laplacians for increasing $N$.  
\item Study convergence properties toward continuum spectra.  
\item Analyze spectral density, spectral gaps, and degeneracy patterns.  
\end{enumerate}

\subsection{Stage 3: Topological Correspondence Tests}
\begin{enumerate}
\item Compare degeneracy structures with known Hodge numbers.  
\item Explore correlations between low-lying eigenvalues 
and Euler characteristics.  
\item Test robustness of these correspondences 
under different discretization schemes.  
\end{enumerate}

\subsection{Stage 4: Computational Interpretation}
\begin{enumerate}
\item Define and refine the notion of \emph{computational capacity} $\mathcal{C}(M)$ 
as a function of spectral invariants.  
\item Interpret $\mathcal{C}(M)$ in terms of possible quantum state encodings.  
\item Explore the analogy with qubits and propose CYbits 
as computational primitives.  
\end{enumerate}

\subsection{Stage 5: Toward Meta-CYbits}
\begin{enumerate}
\item Extend the framework to higher-dimensional Calabi--Yau 3-folds, 
such as the quintic hypersurface in $\mathbb{CP}^4$.  
\item Study spectral stability under deformations of complex structure.  
\item Define Meta-CYbits as composite structures built from families of CYbits.  
\end{enumerate}

\section{Expected Impact}
Confirming these hypotheses would establish 
a quantitative link between CY topology and computational capacity. 
This would form the first step in a larger program 
leading to CYbits and Meta-CYbits, 
providing a foundation for a new class of meta-quantum computation models.  

\section*{Acknowledgements}
The author thanks the VOSCOM Research Initiative 
for providing the conceptual framework.

\begin{thebibliography}{9}
\bibitem{candelas1985}
Candelas et al., ``Vacuum configurations for superstrings,'' 
Nucl. Phys. B, 1985.
\bibitem{exner2008}
Exner, P., ``Quantum graphs: An introduction,'' 
Annals of Physics, 2008.
\bibitem{belhaj2014}
Belhaj, A. et al., ``Qubits from Calabi--Yau manifolds and toric geometry,'' 
arXiv:1408.3952, 2014.
\bibitem{donaldson2005}
Donaldson, S.K., ``Numerical results on Calabi--Yau metrics,'' 
Class. Quantum Grav., 2005.
\bibitem{chung1997}
Chung, F., ``Spectral Graph Theory,'' 
CBMS Regional Conference Series in Mathematics, 1997.
\end{thebibliography}

\section*{Citation (BibTeX - EN)}
\begin{verbatim}
@misc{CY_meta_quantum_2025,
  author       = {Evgeny Monakhov and LCC "VOSCOM ONLINE" Research Initiative},
  title        = {Meta-Quantum Computing on Calabi--Yau Manifolds},
  year         = {2025},
  publisher    = {Zenodo},
  doi          = {10.5281/zenodo.17050352},
  url          = {https://doi.org/10.5281/zenodo.17050352}
  orcid		   = {0009-0003-1773-5476}
  url_orcid     = {https://orcid.org/0009-0003-1773-5476}
  organization = {https://voscom.online/}
}
}
\end{verbatim}


\end{document}
