\documentclass[12pt,a4paper]{article}
\usepackage[utf8]{inputenc}
\usepackage{amsmath,amssymb}
\usepackage{hyperref}
\usepackage{geometry}
\usepackage{verbatim} % for citation block
\geometry{margin=2.5cm}

\title{Meta-Quantum Computing on Calabi--Yau Manifolds. Lecture.\\
From Qudits to CYbits and Meta-CYbits}
\author{Evgeny Monakhov \\ LCC "VOSCOM ONLINE" Research Initiative \\ https://orcid.org/0009-0003-1773-5476}
\date{September 2025}

\begin{document}
\maketitle

\section*{Introduction}
Good afternoon, colleagues. Today I will present a concept that extends the 
current understanding of quantum computing. 
We begin with the familiar --- the bit and the qubit --- and then move step by step 
towards a new computational unit: the \emph{CYbit}, based on Calabi--Yau 
manifolds. We will then take another step and introduce the \emph{Meta-CYbit} --- 
an object that lives in the space of wave functionals. 
In this way, we obtain a system where quantum principles are 
“built into” the very structure of computation.

\section{From Bit to Qubit}
The classical bit is defined as
\[
x \in \{0,1\}.
\]
A qubit is described as a superposition:
\[
|\psi\rangle = \alpha|0\rangle + \beta|1\rangle, \qquad 
|\alpha|^2 + |\beta|^2 = 1.
\]
$n$ qubits span a Hilbert space of dimension $2^n$.

\section{Qudits}
The generalization of a qubit to $d$ dimensions:
\[
|\psi\rangle = \sum_{i=1}^d \alpha_i |i\rangle, 
\qquad \sum_{i=1}^d |\alpha_i|^2 = 1.
\]
Qudits allow storing more information in a single cell 
and reduce the depth of quantum circuits.

\section{CYbit}
Let $M$ be a Calabi--Yau manifold of complex dimension $k$.
We define a CYbit as a section:
\[
\psi: M \to \mathbb{C}^d, \quad \psi(p)=(\psi_1(p),\dots,\psi_d(p)).
\]
The inner product:
\[
\langle \psi,\phi \rangle = \int_M \psi^\dagger(p)\phi(p)\, d\mu(p).
\]
Thus,
\[
\mathcal{H}_{CY} = L^2(M,\mathbb{C}^d).
\]

\section{CYlink and CYgluon}
The interaction between CYbits:
\[
H = \sum_{(p,q)\in E} w_{pq} e^{i\phi_{pq}} 
U_{pq} \otimes |p\rangle\langle q| + \text{h.c.}
\]
Parameters:
\begin{itemize}
\item $w_{pq}$ --- link weight (CY metric),
\item $\phi_{pq}$ --- phase (Berry integral),
\item $U_{pq}$ --- transition operator.
\end{itemize}
We also introduce the \emph{CYgluon} --- an object describing link interactions:
\[
\hat{\mathcal{G}}(L_1,L_2) \sim g \int \Psi[\psi]\,F[L_1,L_2;\psi]\,\mathcal{D}\psi.
\]

\section{Meta-CYbit}
The key step is \emph{second quantization}.
A CYbit lives in $\mathcal{H}_{CY}$.
Now we introduce a \emph{wave functional}:
\[
\Psi[\psi] \in L^2(\mathcal{H}_{CY},\mathcal{D}\psi).
\]
Normalization:
\[
\int |\Psi[\psi]|^2 \,\mathcal{D}\psi = 1.
\]
The evolution is described by a functional Schrödinger equation:
\[
i\hbar \frac{\partial}{\partial t} \Psi[\psi] 
= \hat{\mathcal{H}} \Psi[\psi].
\]

\section{Comparison of Levels}
\begin{tabular}{|c|c|c|}
\hline
Level & Object & State Space \\
\hline
0 & Bit & $\{0,1\}$ \\
1 & Qubit/Qudit & $\mathbb{C}^d$ \\
2 & CYbit & $\mathcal{H}_{CY} = L^2(M,\mathbb{C}^d)$ \\
3 & Meta-CYbit & $\mathcal{H}_{Meta} = L^2(\mathcal{H}_{CY},\mathcal{D}\psi)$ \\
\hline
\end{tabular}

\section{Computational Potential}
Approximate estimates:
\begin{itemize}
\item Classical PC: $10^{12}$ operations/s.
\item 50 qubits: $\approx 10^{15}$ amplitudes.
\item 10 CYbits (3D, $m=10$): $\approx 10^{30}$ amplitudes.
\item 10 CYbits (6D, $m=10$): $\approx 10^{60}$ amplitudes.
\item Meta-CYbit: super-exponential functional space.
\end{itemize}

\section{Implications}
\textbf{Mathematics:} computational geometry of Calabi--Yau. \\
\textbf{Physics:} new applications of CY beyond cosmology. \\
\textbf{Computer Science:} a new paradigm of meta-quantum computing. \\
\textbf{AI:} potential new architectures for learning and modeling.

\section*{Conclusion}
We have traced the path: bit $\to$ qubit $\to$ qudit $\to$ CYbit $\to$ Meta-CYbit.  
This hierarchy opens a new class of computational models, 
where quantum principles are embedded into the very structure of the architecture. 
Further research may lead to devices that operate in spaces 
we are only beginning to explore.

\section*{Citation (BibTeX)}
\begin{verbatim}
@misc{CY_meta_quantum_2025,
  author       = {Evgeny Monakhov and LCC "VOSCOM ONLINE" Research Initiative},
  title        = {Meta-Quantum Computing on Calabi--Yau Manifolds},
  year         = {2025},
  publisher    = {Zenodo},
  doi          = {10.5281/zenodo.17050352},
  url          = {https://doi.org/10.5281/zenodo.17050352}
  orcid		   = {0009-0003-1773-5476}
  url_orcid     = {https://orcid.org/0009-0003-1773-5476}
  organization = {https://voscom.online/}
}
\end{verbatim}

\end{document}
