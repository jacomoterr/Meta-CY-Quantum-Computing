\documentclass[12pt,a4paper]{article}
\usepackage[utf8]{inputenc}
\usepackage{amsmath,amssymb}
\usepackage{geometry}
\geometry{margin=2.5cm}
\usepackage{hyperref}

\title{Research Roadmap for Meta-CY Quantum Computing}
\author{Evgeny Monakhov \\ LCC "VOSCOM ONLINE" Research Initiative \\ 
\href{https://orcid.org/0009-0003-1773-5476}{ORCID: 0009-0003-1773-5476}}
\date{2025}

\begin{document}
\maketitle

\section*{Introduction}
This roadmap outlines the development of the Meta-CY Quantum Computing framework,
where information carriers are defined as quantum states on Calabi--Yau (CY) manifolds,
and operations involve spectral, topological, and functional transformations.  

\section{Main Research Directions}

\subsection{CY-Time: Ricci Flow as Computation}
\begin{itemize}
\item Model:
\[
\frac{\partial g_{i\bar{j}}}{\partial \tau} = - \mathrm{Ric}(g)_{i\bar{j}}.
\]
\item Goal: describe time as projection of internal CY dynamics.  
\item Perspective: predictive models of evolution.  
\end{itemize}

\subsection{Error-Correcting Codes from CY Homology}
\begin{itemize}
\item Code model:
\[
[[n,k,d]], \quad d = \min\{\text{size of nontrivial cycle}\}.
\]
\item Goal: construct new families of robust quantum codes.  
\item Perspective: generalization of toric codes to higher-dimensional CYs.  
\end{itemize}

\subsection{Spectral Properties of CY Graphs}
\begin{itemize}
\item Discrete Laplacian:
\[
(Lf)(p_i) = \sum_{j:(i,j)\in E} w_{ij}(f(p_i)-f(p_j)).
\]
\item Hypothesis: spectrum encodes Hodge numbers $(h^{1,1},h^{2,1})$ and Euler characteristic $\chi(M)$.  
\item Perspective: new link between spectral geometry and algebraic topology.  
\end{itemize}

\subsection{Computational Capacity of CY Manifolds}
\begin{itemize}
\item Definition:
\[
\mathcal{C}(M) = \log \dim_{\mathrm{eff}}(\mathcal{H}_{CY}).
\]
\item Hypothesis:
\[
\mathcal{C}(M) \sim f(h^{1,1}, h^{2,1}, \chi(M)).
\]
\item Perspective: quantitative theory of computational power of CYbits.  
\end{itemize}

\subsection{CYlinks and CYgluons}
\begin{itemize}
\item CYlink:
\[
H_{link}(i,j) = w_{ij}\,\psi^\dagger(p_i)\psi(p_j) + h.c.
\]
\item CYgluon:
\[
H_{gluon}((i,j),(k,l)) = g_{ijkl}\,\psi^\dagger(p_i)\psi(p_j)\psi^\dagger(p_k)\psi(p_l).
\]
\item Perspective: novel architectures of computation with multi-level interactions.  
\end{itemize}

\subsection{Meta-CYbits as Functionals}
\begin{itemize}
\item Definition:
\[
\Psi \in L^2(\mathcal{H}_{CY}, \mathcal{D}\psi).
\]
\item Goal: describe computation at the level of spaces of states.  
\item Perspective: hypercomputation beyond BQP.  
\end{itemize}

\subsection{Mirror Symmetry as Computational Duality}
\begin{itemize}
\item Goal: formalize mirror symmetry as transformation of computations.  
\item Perspective: dual algorithms and translation of problems across regimes.  
\end{itemize}

\subsection{Emergent Phases in CY Quantum Networks}
\begin{itemize}
\item Goal: investigate collective behavior in CYbit networks.  
\item Perspective: emergence of computational phases (analogs of Bose condensation or superconductivity).  
\end{itemize}

\section{Conclusion}
The proposed roadmap integrates mathematical and computational aspects.
Its realization will provide a rigorous theory of Meta-CY Quantum Computing
and uncover new architectures with unique stability and computational capacity.  

\end{document}
