\documentclass[12pt,a4paper]{article}
\usepackage[utf8]{inputenc}
\usepackage{amsmath,amssymb}
\usepackage{geometry}
\geometry{margin=2.5cm}
\usepackage{hyperref}

\title{CYlinks and CYgluons: Interactions in Meta-Quantum Computing}
\author{Evgeny Monakhov \\ LCC "VOSCOM ONLINE" Research Initiative \\ 
\href{https://orcid.org/0009-0003-1773-5476}{ORCID: 0009-0003-1773-5476}}
\date{2025}

\begin{document}
\maketitle

\begin{abstract}
We propose a formal description of \emph{CYlinks} and \emph{CYgluons}, 
the interaction structures in the Meta-CY Quantum Computing framework.  
CYlinks represent couplings between CYbits, defined by embeddings into 
Calabi--Yau (CY) subspaces, while CYgluons describe higher-order 
interactions between the links themselves, analogous to gluons in quantum chromodynamics (QCD).  
We present mathematical definitions, operator formalisms, 
research roadmap, and perspectives for computational applications.  
\end{abstract}

\section{Introduction}
In the Meta-CY Quantum Computing framework, CYbits are quantum states defined on CY manifolds.  
To build computation, CYbits must interact.  
These interactions are not limited to pairwise couplings, but may themselves form 
networks of higher-order dynamics.  
We call these structures CYlinks (direct connections) and CYgluons (link-link interactions).  

\section{Definition of CYlinks}
Let $M$ be a CY manifold, discretized as a graph $G=(V,E)$.  
For two CYbits at points $p_i,p_j \in M$, define a CYlink as an operator
\begin{equation}
H_{link}(i,j) = w_{ij} \, \psi^\dagger(p_i)\psi(p_j) + h.c.,
\end{equation}
with weight
\begin{equation}
w_{ij} = f(\mathrm{dist}_M(p_i,p_j), \;\mathcal{T}_{ij}),
\end{equation}
where $\mathcal{T}_{ij}$ encodes topological data of the embedding subspace.

\subsection{Graph Laplacian Form}
The total link Hamiltonian can be expressed as:
\begin{equation}
H_{links} = \sum_{(i,j)\in E} w_{ij} ( \psi^\dagger(p_i)\psi(p_j) + h.c. ).
\end{equation}
This generalizes standard adjacency-based interactions to CY-dependent weights.

\section{CYgluons: Interactions Between Links}
CYlinks themselves may interact, forming higher-order couplings.  
Define a CYgluon operator acting on two links $(i,j)$ and $(k,l)$:
\begin{equation}
H_{gluon}((i,j),(k,l)) = g_{ijkl} \, \psi^\dagger(p_i)\psi(p_j)\psi^\dagger(p_k)\psi(p_l),
\end{equation}
where
\begin{equation}
g_{ijkl} = g(\mathcal{T}_{ij}, \mathcal{T}_{kl}, \; \mathrm{Hom}(M)).
\end{equation}
Here $g_{ijkl}$ depends on overlaps of CY subspaces and homological relations.  

\subsection{Total Hamiltonian}
The global system Hamiltonian is then
\begin{equation}
H = \sum_i H_{CYbit}(i) + \sum_{(i,j)} H_{link}(i,j) + \sum_{(i,j),(k,l)} H_{gluon}((i,j),(k,l)).
\end{equation}

\section{Research Roadmap}
\subsection{Stage I: CYlink Formalism}
\begin{enumerate}
\item Define explicit $w_{ij}$ for simple CY (tori $T^n$, K3).  
\item Compute spectra of CYlink Hamiltonians.  
\item Relate $w_{ij}$ to CY topology.  
\end{enumerate}

\subsection{Stage II: CYgluon Structures}
\begin{enumerate}
\item Introduce $g_{ijkl}$ based on overlaps of CY subspaces.  
\item Test consistency with gauge-like symmetries.  
\item Explore analogy with QCD color charges.  
\end{enumerate}

\subsection{Stage III: Combined Dynamics}
\begin{enumerate}
\item Simulate CYbit networks with CYlinks and CYgluons.  
\item Study stability and error correction properties.  
\item Explore emergent computational phases.  
\end{enumerate}

\section{Perspectives}
\begin{itemize}
\item \textbf{Computational Power:} CYlinks and CYgluons provide new interaction structures, potentially enhancing expressive capacity.  
\item \textbf{Error Correction:} CYgluon couplings may stabilize logical states via higher-order redundancies.  
\item \textbf{Physics Analogy:} Formal similarity with gauge field theories suggests extensions to CY gauge groups.  
\end{itemize}

\section{Conclusion}
CYlinks and CYgluons extend the Meta-CYbit model by introducing structured interactions 
between CYbits and between their couplings.  
This offers a foundation for novel quantum computational architectures 
grounded in Calabi--Yau geometry.  

\section*{Citation (BibTeX - EN)}
\begin{verbatim}
@misc{CY_links_gluons_2025,
  author       = {Evgeny Monakhov and LCC "VOSCOM ONLINE" Research Initiative},
  title        = {CYlinks and CYgluons: Interactions in Meta-Quantum Computing},
  year         = {2025},
  publisher    = {Zenodo},
  doi          = {10.5281/zenodo.17050353},
  url          = {https://doi.org/10.5281/zenodo.17050353},
  orcid        = {0009-0003-1773-5476},
  url_orcid    = {https://orcid.org/0009-0003-1773-5476},
  organization = {https://voscom.online/}
}
\end{verbatim}

\end{document}
